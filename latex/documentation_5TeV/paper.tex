\documentclass[aps,prd,final,twocolumn,letterpaper]{revtex4}
\usepackage[latin1]{inputenc}
\usepackage{graphicx}
\usepackage{physics}
\usepackage{siunitx}
\usepackage{amsmath}

%%%%%% Document Begins Here %%%%%%%%

\begin{document}
	\title{Measuring the inclusive cross section for W-boson and Z-boson production in pp collisions at $\sqrt{s}=5\si{\tera\electronvolt}$ with the CMS detector at the LHC}
	\author{Alexander~Andriatis}
	\affiliation{ Particle Physics Collaboration, MIT Department of Physics,
		77 Massachusetts Ave.,
		Cambridge, MA 02139}
	\date{\today} 
	
	\begin{abstract}
		\noindent	
		The inclusive cross section of vector boson production in proton-proton collisions at the LHC is one of the key measurements for constraining the standard model. The measurement at $\sqrt{s}=5\si{\tera\electronvolt}$ is reported. The analysis includes systematic uncertainties from theoretical predictions as well as detector performance effects. A fit on the combined results from different center of mass energies is used to further constrain PDFs.
	\end{abstract}
	
	\maketitle
	
	\tableofcontents
	
	\pagestyle{myheadings}
	\markboth{A. Andriatis}{Inclusive W and Z cross section at $\sqrt{s}=5\si{\tera\electronvolt}$ }
	\thispagestyle{empty}
	
	\section{Introduction}	
		The inclusive cross section of vector boson production in proton-proton collisions at the LHC is one of the key measurements for constraining the standard model. The measurement at $\sqrt{s}=5\si{\tera\electronvolt}$ is reported. The analysis includes systematic uncertainties from theoretical predictions as well as detector performance effects. A fit on the combined results from different center of mass energies is used to further constrain PDFs.
	
	\section{Objects}
	\section{Data Samples}
	\section{Event Selection}
	\section{Signal Extraction}
	\section{Detector Efficiency}
	\section{Luminosity}
	\section{Theoretical Uncertainty}
		\subsection{Acceptances}
			\subsubsection{Generator-Level Acceptance}
			The detector acceptance is an important theoretical value for the extraction of the inclusive cross-section from the measured fiducial cross-section. The acceptance is defined as the ratio of events whose decay products fall within the kinematic acceptance region of the detector to the total number of events. As a baseline, events are simulated using the next-to-leading-order Monte Carlo aMC@NLO with NNPDF3.0 parton distribution function set and PYTHIA 8 parton showering and hadronization. The error on the simulated distribution is estimated by comparing the acceptances to those derived using higher-order tools, such as RESBOS and DYRES.
			
			The fiducial region of the detector accepts only muons of $p_{T} > 25\si{\giga\electronvolt}$ and $\abs{\eta} < 2.4$ and electrons with $p_{T} > 25\si{\giga\electronvolt}$ and $\abs{\eta} < 1.4442$ or $1.566 < \abs{\eta} < 2.5$. An additional constraint is imposed on the Z channel that $60\si{\giga\electronvolt} < m_{z} < 120\si{\giga\electronvolt}$.
			
			The generator-level acceptance is considered for post-FSR particles for several decay channels. A summary of the acceptances and their statistical uncertainty is given in table \ref{tab:gen_acceptance}.
			
			\begin{table}
				\centering
				\begin{tabular}{l  c}
					Process & $A_{Gen}$(Post-FSR) \\
					$Z\rightarrow \mu^{+}\mu^{-}$ & $0.440\pm 0.001$ \\
					$Z\rightarrow e^{+}e^{-}$ & $0.406\pm 0.001$ \\
					$W^{+}\rightarrow \mu^{+}\nu$ & $0.572\pm 0.001$ \\
					$W^{+}\rightarrow e^{+}\nu$ & $0.547\pm 0.001$ \\
					$W^{-}\rightarrow \mu^{-}\bar{\nu}$ & $0.536\pm 0.001$ \\
					$W^{-}\rightarrow e^{-}\bar{\nu}$ & $0.515\pm 0.001$ \\	
				\end{tabular}
				\caption{Gen-level acceptance from aMC@NLO with NNPDF3.0 and Pythia8 for signal processes $Z\rightarrow l^{+}l^{-}$ and $W\rightarrow l\nu$}.
				\label{tab:evolution}
			\end{table}
			
	\section{Results}
		\subsection{Uncertainty Correlations}
		\subsection{Cross-Section Results}
	\section{Summary}

	\bibliography{paper}
	
\end{document}