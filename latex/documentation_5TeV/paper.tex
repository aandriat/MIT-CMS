\documentclass[aps,prd,final,twocolumn,letterpaper]{revtex4}
\usepackage[latin1]{inputenc}
\usepackage{graphicx}
\usepackage{physics}
\usepackage{siunitx}
\usepackage{amsmath}

%%%%%% Document Begins Here %%%%%%%%

\begin{document}
	\title{Measuring the inclusive cross section for W-boson and Z-boson production in pp collisions at $\sqrt{s}=5\si{\tera\electronvolt}$ with the CMS detector at the LHC}
	\author{Alexander~Andriatis}
	\affiliation{ Particle Physics Collaboration, MIT Department of Physics,
		77 Massachusetts Ave.,
		Cambridge, MA 02139}
	\date{\today} 
	
	\begin{abstract}
		\noindent	
		The inclusive cross section of vector boson production in proton-proton collisions at the LHC is one of the key measurements for constraining the standard model. The measurement at $\sqrt{s}=5\si{\tera\electronvolt}$ is reported. The analysis includes systematic uncertainties from theoretical predictions as well as detector performance effects. A fit on the combined results from different center of mass energies is used to further constrain PDFs.
	\end{abstract}
	
	\maketitle
	
	\tableofcontents
	
	\pagestyle{myheadings}
	\markboth{A. Andriatis}{Inclusive W and Z cross section at $\sqrt{s}=5\si{\tera\electronvolt}$ }
	\thispagestyle{empty}
	
	\section{Introduction}	
		The inclusive cross section of vector boson production in proton-proton collisions at the LHC is one of the key measurements for constraining the standard model. The measurement at $\sqrt{s}=5\si{\tera\electronvolt}$ is reported. The analysis includes systematic uncertainties from theoretical predictions as well as detector performance effects. A fit on the combined results from different center of mass energies is used to further constrain PDFs.
	
	\section{Objects}
	\section{Data Samples}
	\section{Event Selection}
	\section{Signal Extraction}
	\section{Detector Efficiency}
	\section{Luminosity}
	\section{Theoretical Uncertainty}
		\subsection{Acceptances}
			\subsubsection{Generator-Level Acceptance}
			The detector acceptance is an important theoretical value for the extraction of the inclusive cross-section from the measured fiducial cross-section. The acceptance is defined as the ratio of events whose decay products fall within the kinematic acceptance region of the detector to the total number of events.
			
			As a baseline, events are simulated using the next-to-leading-order Monte Carlo aMC@NLO with NNPDF3.0 parton distribution function set and PYTHIA 8 parton showering and hadronization. The error on the simulated distribution is estimated by comparing the acceptances to those derived using higher-order tools, such as RESBOS and DYRES.
			
			The fiducial region of the detector accepts muons of $p_{T} > 25\si{\giga\electronvolt}$ and $\abs{\eta} < 2.4$ and electrons with $p_{T} > 25\si{\giga\electronvolt}$ and $\abs{\eta} < 1.4442$ or $1.566 < \abs{\eta} < 2.5$. Z boson events are defined as having the mass of the Z $60\si{\giga\electronvolt} < m_{z} < 120\si{\giga\electronvolt}$.
			
			The generator-level acceptance is considered for post-FSR particles for several decay channels and their ratios. A summary of the acceptances and their statistical uncertainty is given in table \ref{tab:gen_acceptance_electrons} for electronic final states and table \ref{tab:gen_acceptance_muons} for muonic final states.
			
			\begin{table}
				\centering
				\begin{tabular}{l  c}
					Process 							& $A_{Gen}$(Post-FSR) 		\\
					$W^{+}\rightarrow e^{+}\nu$ 		& $0.430 \pm 0.002$ 		\\
					$W^{-}\rightarrow e^{-}\bar{\nu}$ 	& $0.515\pm 0.001$ 			\\		
					$W\rightarrow e\nu$ 				& $0.515\pm 0.001$ 			\\
					$W^{+}/W^{-}$					 	& $0.515\pm 0.001$ 			\\
					$Z\rightarrow e^{+}e^{-}$ 			& $0.406\pm 0.001$ 			\\
					$W^{+}/Z$				 			& $0.406\pm 0.001$ 			\\
					$W^{-}/Z$				 			& $0.406\pm 0.001$ 			\\
					$W/Z$				 				& $0.406\pm 0.001$ 			\\					
				\end{tabular}
				\caption{Gen-level acceptance from aMC@NLO with NNPDF3.0 and Pythia8 for electronic final states}.
				\label{tab:gen_acceptance_electrons}
			\end{table}

			\begin{table}
				\centering
				\begin{tabular}{l  c}
					Process 								& $A_{Gen}$(Post-FSR) 		\\
					$W^{+}\rightarrow \mu^{+}\nu$ 			& $0.430 \pm 0.002$ 		\\
					$W^{-}\rightarrow \mu^{-}\bar{\nu}$ 	& $0.515\pm 0.001$ 			\\		
					$W\rightarrow \mu\nu$ 					& $0.515\pm 0.001$ 			\\
					$W^{+}/W^{-}$					 		& $0.515\pm 0.001$ 			\\
					$Z\rightarrow \mu^{+}\mu^{-}$ 			& $0.406\pm 0.001$ 			\\
					$W^{+}/Z$				 				& $0.406\pm 0.001$ 			\\
					$W^{-}/Z$				 				& $0.406\pm 0.001$ 			\\
					$W/Z$				 					& $0.406\pm 0.001$ 			\\					
				\end{tabular}
				\caption{Gen-level acceptance from aMC@NLO with NNPDF3.0 and Pythia8 for muonic final states}.
				\label{tab:gen_acceptance_muons}
			\end{table}
			
			To compare the theoretical acceptance between 5 TeV and 13 TeV, for each process the ratio of the acceptances was calculated. The acceptance ratio of 5 TeV / 13 TeV is reported in table \ref{tab:pdf_ratios_electrons} for electronic final states and \ref{tab:pdf_ratios_muons} for muonic final states.
			
			\subsubsection{PDF Uncertainty}
			The parton distribution functions (PDFs) used to describe the kinematics of the constituents in the colliding protons have inherent theoretical uncertainty. Different groups have calculated PDFs with a variety of methods, using both data and theoretical predictions. The acceptances for the nominal values from each PDF set are shown in table \ref{tab:pdf_comparison}.
			
			Each PDF set also provides a measure of its uncertainty. The baseline amc@NLO uses NNPDF3.0, which provides a set of 100 replica PDFs from which we calculate the acceptance error as the the standard deviation of acceptances across the replica sets. 
			
			A contribution to the uncertainty of the PDF set is the value of $\alpha_{s}$ used in making the PDF set. The PDF4LHC working group recommends calculating the total uncertainty from the PDF by summing in quadrature the uncertainties from the PDF and Alphas contributions. 
			
			For the case of NNPDF provides a set of PDF replicas. The gen-level acceptance is calculated for each replica PDF, and the standard deviation of acceptances is reported as the PDF uncertainty on each acceptance. The results are summarized in table \ref{tab:pdf_uncertainty_electrons} for electronic final states and \ref{tab:pdf_uncertainty_muons} for muonic final states.
			
			\begin{table}
				\centering
				\begin{tabular}{l  c}
					Process 							& $\sigma_{A_{Gen}}$(Post-FSR) 		\\
					$W^{+}\rightarrow e^{+}\nu$ 		& $0.430 \pm 0.002$ 		\\
					$W^{-}\rightarrow e^{-}\bar{\nu}$ 	& $0.515\pm 0.001$ 			\\		
					$W\rightarrow e\nu$ 				& $0.515\pm 0.001$ 			\\
					$W^{+}/W^{-}$					 	& $0.515\pm 0.001$ 			\\
					$Z\rightarrow e^{+}e^{-}$ 			& $0.406\pm 0.001$ 			\\
					$W^{+}/Z$				 			& $0.406\pm 0.001$ 			\\
					$W^{-}/Z$				 			& $0.406\pm 0.001$ 			\\
					$W/Z$				 				& $0.406\pm 0.001$ 			\\					
				\end{tabular}
				\caption{Uncertainty on acceptance values for aMC@NLO with NNPDF3.0 and Pythia8 for electronic final states}.
				\label{tab:pdf_uncertainty_electrons}
			\end{table}
			
			\begin{table}
				\centering
				\begin{tabular}{l  c}
					Process 								& $\sigma_{A_{Gen}}$(Post-FSR) 		\\
					$W^{+}\rightarrow \mu^{+}\nu$ 			& $0.430 \pm 0.002$ 		\\
					$W^{-}\rightarrow \mu^{-}\bar{\nu}$ 	& $0.515\pm 0.001$ 			\\		
					$W\rightarrow \mu\nu$ 					& $0.515\pm 0.001$ 			\\
					$W^{+}/W^{-}$					 		& $0.515\pm 0.001$ 			\\
					$Z\rightarrow \mu^{+}\mu^{-}$ 			& $0.406\pm 0.001$ 			\\
					$W^{+}/Z$				 				& $0.406\pm 0.001$ 			\\
					$W^{-}/Z$				 				& $0.406\pm 0.001$ 			\\
					$W/Z$				 					& $0.406\pm 0.001$ 			\\					
				\end{tabular}
				\caption{Uncertainty on acceptance values for aMC@NLO with NNPDF3.0 and Pythia8 for muonic final states}.
				\label{tab:pdf_uncertainty_muons}
			\end{table}
			
			The uncertainty on the acceptance ratios between 5 TeV and 13 TeV is given in table \ref{tab:pdf_ratios_electrons} for electronic final states and \ref{tab:pdf_ratios_muons} for muonic final states 
			
			\begin{table}
				\centering
				\begin{tabular}{l c c} 
					Process 							& $A_{5TeV}/A_{13TeV}$ 		&$\sigma_{A_{5TeV}/A_{13TeV}}$	\\
					$W^{+}\rightarrow e^{+}\nu$ 		& $0.430 \pm 0.002$ 		& $0.430 \pm 0.002$ \\
					$W^{-}\rightarrow e^{-}\bar{\nu}$ 	& $0.515\pm 0.001$ 			& $0.430 \pm 0.002$ \\		
					$W\rightarrow e\nu$ 				& $0.515\pm 0.001$ 			& $0.430 \pm 0.002$ \\
					$W^{+}/W^{-}$					 	& $0.515\pm 0.001$ 			& $0.430 \pm 0.002$ \\ 
					$Z\rightarrow e^{+}e^{-}$ 			& $0.406\pm 0.001$ 			& $0.430 \pm 0.002$ \\
					$W^{+}/Z$				 			& $0.406\pm 0.001$ 			& $0.430 \pm 0.002$ \\
					$W^{-}/Z$				 			& $0.406\pm 0.001$ 			& $0.430 \pm 0.002$ \\
					$W/Z$				 				& $0.406\pm 0.001$ 			& $0.430 \pm 0.002$ \\					
				\end{tabular}
				\caption{Acceptance ratios and their uncertainties for 5TeV acceptance / 13 TeV acceptance from aMC@NLO with NNPDF3.0 and Pythia8 for electronic final states}.
				\label{tab:pdf_ratios_electrons}
			\end{table}
			
			\begin{table}
				\centering
				\begin{tabular}{l c c}
					Process 								& $A_{5TeV}/A_{13TeV}$ 		&$\sigma_{A_{5TeV}/A_{13TeV}}$	\\
					$W^{+}\rightarrow \mu^{+}\nu$ 			& $0.430 \pm 0.002$ 		& $0.430 \pm 0.002$ \\
					$W^{-}\rightarrow \mu^{-}\bar{\nu}$ 	& $0.515\pm 0.001$ 			& $0.430 \pm 0.002$ \\		
					$W\rightarrow \mu\nu$ 					& $0.515\pm 0.001$ 			& $0.430 \pm 0.002$ \\
					$W^{+}/W^{-}$					 		& $0.515\pm 0.001$ 			& $0.430 \pm 0.002$ \\
					$Z\rightarrow \mu^{+}\mu^{-}$ 			& $0.406\pm 0.001$ 			& $0.430 \pm 0.002$ \\
					$W^{+}/Z$				 				& $0.406\pm 0.001$ 			& $0.430 \pm 0.002$ \\
					$W^{-}/Z$				 				& $0.406\pm 0.001$ 			& $0.430 \pm 0.002$ \\
					$W/Z$				 					& $0.406\pm 0.001$ 			& $0.430 \pm 0.002$ \\					
				\end{tabular}
				\caption{Acceptance ratios and their uncertainties for 5TeV acceptance / 13 TeV acceptance from aMC@NLO with NNPDF3.0 and Pythia8 for muonic final states}.
				\label{tab:pdf_ratios_muons}
			\end{table}			
	
		
	\section{Results}
		\subsection{Uncertainty Correlations}
		\subsection{Cross-Section Results}
	\section{Summary}

	\bibliography{paper}
	
\end{document}